\chapter{Representation}

This chapter contains a discussion about the possible ways of representing the important properties of the abstraction of the world in which the autonomous vehicle will perform computations in order to achieve its goal.

We will discuss namely the way to represent the map of the world, the racing track, the coordinate system which will define the position of objects in the map, the definition of the racing track, the state space of the vehicle, the space of possible actions, and the way we will consider the temporal nature of the problem.

\section{Map of the World}

For the purposes of localization and navigation we need to have a good representation of the regions which are free and the regions which are occupied by various obstacles (e.g., walls). The map will be used to determine the location of the vehicle based on the readings from the sensors, identifying obstacles, and determining whether certain point of the world is free or occupied so we can avoid any collisions.

We assume that we will be given the map of the racing track in advance and that the robot does not have to explore the world in the first place. During the race, there still might be some unexpected obstacles which have to be detected. We can use an implementation of the two dimensional \textit{Simultaneous Localization and Mapping} (SLAM) algorithm \cite{SLAM} to preform this task for us.

\paragraph{SLAM} collects the readings from various sensors of a robot (e.g., distance to the nearest landmarks obtained from a laser scan, odometry data) and compares this information to the previous known location of the robot and the map which has been created so far. The laser scan is compared to the map which was compiled so up to this point. A transformation (rotation and translation) will be found to overlap with the known map and the precise location of the robot. The odometric data are used to estimate the new position of the robot and assist in the searching for the best transformation. The map can then be extended in the parts where previous readings were missing (e.g., when the robot turns around a corner) and improved in the regions where the laser scan readings are overlapping with the previous readings.

% todo: Add an image of the matching step

\begin{subparagraph}
For each point of the world we should be able to say what is its status. Points which have not been examined yet will be marked as \textit{unknown}. All of the already examined points will be assigned a probability with which the point is inside an obstacle region. The value of 0 will mark a certainly free space and a value of 1 a certain obstacle. Values between 0 and 1 can be caused by conflicting readings (e.g., a region where there was an obstacle, but it was removed or vice versa) or due to an error in the sensor reading or the fact that the point is close to the maximum range of the sensor and the reading is not considered very reliable.
\end{subparagraph}

\paragraph{Occupancy Grid} is a data structure which represents a map with the properties described in the previous paragraph of a fixed width and height and with fixed horizontal and vertical resolution. It is essentially a two dimensional array of numbers representing whether the fact that the state is unknown (e.g., $-1$) or the probability of an obstacle (e.g., a real number from the interval of $[0; 1]$ or an integer from some range like $[0; 100]$. The advantage of this data structures are the fast test of whether the certain point is occupied or not which is useful both for finding the best transformation of the laser scan during SLAM or for testing collisions in the planning algorithm for verifying if a certain action can be used or not.

% todo: Add an image of an example of an occpupancy grid visualisation

\begin{subparagraph}
The occupancy grid data structure is a widely used data structure by various SLAM libraries like Hector SLAM \cite{hector}, gmapping \cite{gmapping}, or Google Cartographer \cite{google_cartographer}. We will therefore use the occupancy grid as a representation of the world map in this thesis.
\end{subparagraph}

For the sake of simplicity we assume that the map we will be given at the start of the race will contain precise information of the racing track marked as certainly free regions and a continuous border of the race track marked as a certain obstacle. The map will not include obstacles placed along the track.

\section{Racing Track}

Prior to the start of a race we must be given a sequence of points in the free regions of the map. Straight lines between two consecutive points should always lay inside of a free region of the map (but there might be an obstacle which is not marked in the map). The order of these points defines the direction in which the track should be driven through.

The vehicle does not have to travel along the lines between those points nor does it have to hit any of these points with the exception of the first and the last point.

The vehicle will be placed at the location of the first point and the orientation of the vehicle will be the same as the angle between the first two points. The last point of the sequence marks the finishing point of the race track. The vehicle should come to a halt at this point.

To achieve a racing track which consists of several loops, a number of laps can be given. It is then assumed that the sequence should be repeated this number of times and the very first point will be considered to be also the goal point.

\section{Coordinate System}

The most straightforward way of representing the position of a vehicle in the 2D map is to use the Cartesian coordinate system - each point of the plane will be represented by a two dimensional vector of real numbers denoting the distance along the $x$ and $y$ from a defined origin.

A different approach would be for example to find a reference path (e.g., the center line of the track) along the racing track and parametrize the track as a curvature profile that would be a function of the distance along the path. Additionally we would store the distance to the inside and outside tack boundaries along the reference path. This method was used in a dissertation thesis by Nitin R. Kapania \cite{dissertation} successfully for fast generation path planning.

For the purposes of this thesis we will use the Cartesian coordinate system which will provide sufficient means for path planning and it will be easy to use these coordinates to check for collisions with obstacles in the occupancy grid.

\section{State Space}

We can represent this state of the vehicle with a configuration $( \, \vec{x}, \vec{v}, \alpha, \theta ) \,$, where $\vec(x)$ is the position vector, $\vec{v}$ is the vector of the instantaneous velocity, $\alpha$ is the steering angle of the front wheels, and $\theta$ is the orientation of the vehicle.

The difference between the orientation of the vehicle and the angle of the velocity is called the slip angle. We assume that only the front wheels can be steered as it is common for most car-like vehicles.

\section{Vehicle Control Actions}

To control the vehicle, we must be able to to control its longitudal and lateral velocities. Robots can be divided into two basic classes of holonomic and non-holonomic driving robots.

\paragraph{Holonomic} drive robots have the number of degrees of freedom of movement equal to the total number of the degrees of freedom \cite{}. A robot with holonomic drive can move in any direction it needs to. An example of holonomic drive can be robots with omni-wheels or castor wheels.

\paragraph{Non-holonomic} drive robots on the other hand have a smaller number of degrees of freedom than the total number of degrees of freedom. There are three degrees of freedom for a robot in a 2D environment: its position in the two axes and its orientation. Car-like robots have only two controllable degrees of freedom though: the acceleration (positive or negative) which moves the robot parallel to the direction of its orientation, and changing the angle of its wheels. This limitation results in a more complex turning maneuvers when compared to the holonomic drive robots.

We will assume that our car-like robot has a non-holonomic drive and that it is able to change only the steering angle of its front wheels, not the rear. Most of the personal cars in operation as of writing this thesis are constructed in this way and we would like to test our algorithms on a robot which has similar driving characteristics to the common real world cars.

The vehicle is moving continuously in the environment and potentially at high speeds. Any action we will apply to the vehicle cannot change the velocities instantaneously and directly. Actions will be used to control the actuators of the vehicle and these will create forces which will cause the changes of movement of the vehicle.

A typical car-like vehicle has a way of adjusting the longitudal velocity by applying throttle or by breaking. We assume that the vehicle has only front wheel steering and the rear tires are fixed and always aligned with the orientation vehicle.

We define an action as a pair $( \, t, \phi ) \,$ where $t \in [ \, -1, 1 ] \,$ is the amount of applied throttle ($-1$ is full reverse, $0$ is no throttle, $1$ is full forward throttle) and $\phi$ is the desired steering angle of the front wheels ($-1$ is the maximum angle to the left, $0$ is the orientation of the vehicle, $1$ is the maximum angle to the right). To achieve breaking an action with the required amount of throttle with an opposite sign to the direction of motion should be used.

When this abstraction is applied to a specific hardware platform it should be possible to reduce the number of the actions because of hardware limitations. The actuators do not have infinite resolution and thus the $( \, -1, 1 ) \,$ interval can be divided into a constant number of subintervals which correspond to the same amount of throttle or the same steering angle. This means that the number of actions is reduced to $mn$ where $m$ is the number of different steering angle and $n$ is the number of different throttle levels the actuators can achieve.

This gives us a finite and discrete set of actions we can use to control the motion of the vehicle.

\section{Time Representation}

The actions described in the previous section do not have any intrinsic representation of time. If we created a plan as a sequence of these actions, we would have hard time executing them as we would not know at which time to apply these actions. And the timing of these actions is very important as it has significant implications for the state of the vehicle.

Our actions do not change the state of the vehicle (position, velocity, orientation) directly but through the use of actuators (acceleration, steering angle). By applying the same acceleration for two different time periods, we achieve two different velocities of the vehicle.

We should also consider hardware limitations of the actuators of a vehicle. It takes some time $t_{adj} > 0$ for them to switch from the previous action to the new action. It should be obvious that this time period is not constant and will differ for different pairs of actions based on the amount of change in the. We can imagine this as the time it would take a human driver to turn the steering wheel and press the accelerator paddle into the desired position.

The first way how to deal with this problem would be to add another parameter $t \in ( \, t_{adj}, \infinity) \,$ which would describe the time period in seconds for which the action should be applied.

A second way might be to choose a constant time period $t_{const} \in ( \, t_{adj}, \infinity) \,$ in seconds for every action. This implicit representation of time would keep the actions simple and reduce the number dimensions of the search space for the planning algorithm. On the other hand we might need to repeat one action several times to achieve the same maneuvers which could be achieved by one longer action (e.g. a U-turn maneuver) and the resulting plans will be longer.

We can address this problem by compressing the plan by merging repeating actions into a single action and a discrete parameter $r \in { \, 2, 3, 4, ..., r_{max} } \,, r_{max} \in \natural$ which will describe how many times the action should be repeated. This will shift our representation of time towards the first approach, but will still keep the action space discrete and finite.

For the sake of simplicity we will use the implicit representation of time with the compression of repeating actions. We will also assume that $t_{adj}$ is close to zero and we will assume that the changes of actions are immediate.