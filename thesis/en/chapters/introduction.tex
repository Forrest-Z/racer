\chapter*{Introduction}
\addcontentsline{toc}{chapter}{Introduction}

Autonomous vehicles are slowly making their way from research facilities to the public roads around the world and it seems inevitable that soon they will become parts of our everyday lives. Autonomous vehicles could have great impact on transportation of people and goods. For example, thousands of people die in car accidents every year because of human error \cite{Road_accidents}. Some believe that replacing humans with accurate electronic sensors and computer algorithms, which never make errors, could reduce the number of accidents on roads significantly. It seems that companies around the world see the potential in this technology and invest a lot of effort and money in research of self-driving cars. Even though nobody can guarantee that autonomous vehicles will never make errors and that they will solve other problems of road transportation we face today, the trend seems to be clear. The key to the success of this technology will be reliable and robust algorithms which will work in all driving scenarios and weather conditions and which the public will trust. 

The idea of self-driving cars is not a new one. Prototypes of autonomous vehicles were demonstrated on public roads already in the 1980s. Some of the well-known projects from this era were for example the NavLab vehicle of the Carnegie Mellon University \cite{NavLab} or the project of Ernst Dickmanns in cooperation with the Daimler company \cite{Ernst_Dickmanns}. The DARPA challenges, organized between 2004 and 2007, gained a lot of media attention. We could see that self-driving cars are not just sci-fi anymore. Since then, many driving assistants such as adaptive cruise control, lane keeping assistants, parallel parking assistants and collision warning and emergency braking assistants started appearing in commercially available vehicles. Some systems, like the Tesla Autopilot, go even further and allow complex autonomous maneuvers such as overtaking of other vehicles.

These commercially available assistants provide only partial autonomy. They require the driver to always pay attention to the road and to be prepared to take over the control of the vehicle. Fully autonomous vehicles will not require any human interaction and they might even lack any means of manual driving input and they might not even have a steering wheel and the pedals. The computer will be responsible for the vehicle from the start to the destination no matter what the weather and traffic is like outside. There is an ongoing research into full autonomy by some companies, including Waymo and Uber, and the prototypes of these vehicles are already being tested on public roads.

One of the key problems in autonomous driving is the ability to find a good trajectory for the car. This trajectory must be safe for the passengers of the vehicle and for other road users. It must also conform to road limitations such as speed limits and other traffic rules. Automated planning is an area of artificial intelligence which gives us means to solve this problem. For a well-defined planning problem which describes the physics of the vehicle and what our goals are, we can find a sequence of control inputs for the given type of vehicle which will drive the vehicle the way we need.

The goal of this thesis is to create an artificial agent which will autonomously drive along a racing circuit utilizing a trajectory planning algorithm. The agent must avoid any collision with the boundary of the track or any static obstacles on the track while trying to achieve fast lap times. We will not consider the option of head-to-head racing where there are multiple competing cars moving on the track at the same time and all the obstacles on the track will be stationary.

We will formulate a trajectory planning problem of driving through a series of waypoints while aiming for fast lap times. This thesis explores ways of finding a solution to this problem with a knowledge of an approximate model of the vehicle. Alongside the planning algorithm, we explore the possibilities of following the selected reference trajectory as closely as possible while avoiding obstacles which were not known at the time of planning. These two algorithms are combined in an autonomous racing agent which uses the trajectory planning algorithm to find a trajectory for the next few corners ahead, and then follows this trajectory. We implement and test this racing behavior on a 1:10 scale car-like robot with a set of on-board sensors and an embedded computer.

First, we introduce the autonomous driving problem and split it into several sub-problems. We give a brief overview of related works in the field of autonomous driving.

Next, we examine how the vehicle reacts to steering commands and we describe two vehicle models with different levels of complexity and accuracy in the chapter Vehicle Model. With a model of the vehicle defined, we formulate the trajectory planning problem for an autonomous racing car, and we describe several methods of finding a feasible plan in the chapter Trajectory Planning Problem. The chapters Trajectory Following and The Racing Agent then combine the previously described algorithms into the final autonomous racing agent.

Finally, we implement the racing agent and we conduct several experiments to verify the performance and success rate of our algorithms on a physical robot in the chapter Experimental Verification, and summarize our findings.
