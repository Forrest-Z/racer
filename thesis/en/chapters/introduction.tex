\chapter*{Introduction}
\addcontentsline{toc}{chapter}{Introduction}

Autonomous vehicles are slowly making their way from research facilities to the public roads around the world and it seems inevitable that they will soon become a part of our everyday lives. Autonomous vehicles could have great impact on transportation of people and goods. For example, thousands of people die in car accidents every year because of human error \cite{Road_accidents}. Some believe that replacing humans with accurate electronic sensors and computer algorithms, which never make errors, could reduce the number of accidents on roads significantly. It seems that companies around the world see the potential in this technology and invest a lot of effort and money in research of self-driving cars. Even though nobody can guarantee that autonomous vehicles will never make errors and that they will solve other problems of road transportation we face today, the trend seems to be clear. The key to the success of this technology will be reliable and robust algorithms which will work in all driving scenarios and weather conditions and which the public will trust.

The idea of self-driving cars is not new. Prototypes of autonomous vehicles were demonstrated on public roads as early as in the 1980s. Some of the well-known projects from this era were for example the NavLab vehicle of the Carnegie Mellon University \cite{NavLab} or the project of Ernst Dickmanns in cooperation with the Daimler company \cite{Ernst_Dickmanns}. The DARPA challenges, organized between 2004 and 2007, gained a lot of media attention. We could see that self-driving cars are not sci-fi anymore. Since then, many driving assistants such as adaptive cruise control, lane keeping assistants, parallel parking assistants, collision warning, and emergency braking assistants started appearing in commercially available vehicles. Some systems, like the Tesla Autopilot, go even further and allow complex autonomous maneuvers such as overtaking of other vehicles.

These commercially available assistants provide the user only with partial autonomy. The driver is required to pay attention to the road at all times and he or she is required to be prepared to take control of the vehicle if the system fails or if it cannot handle the traffic situation. Fully autonomous vehicles will not require any human interaction. They might even lack any means of manual driving. The computer will be responsible for the vehicle from the start to the destination no matter what the weather and traffic is like outside. There is an ongoing research into full autonomy by some companies like Waymo and Uber and prototypes of vehicles equipped with this technology are already being tested on public roads.

One of the key problems in autonomous driving is the ability to find a good trajectory for the car in various. This trajectory must be safe for the passengers of the vehicle and for other road users. It must also conform to road limitations such as speed limits and other traffic rules. Automated planning is an area of artificial intelligence which gives us means to formulate and solve this problem. For a well-defined planning problem which describes the physics of the vehicle and our goals, we can find a sequence of control inputs which will drive the vehicle to the goal as we need.

First, we introduce the autonomous driving problem and split it into several sub-problems. We give a brief overview of related works in the field of autonomous driving. We will describe an architecture of an artificial agent for the task of autonomous racing.

Next, we examine the individual subproblems we must solve in order to create the racing agent. We will study how the vehicle reacts to steering commands and we describe two vehicle models with different levels of complexity and accuracy in the chapter Vehicle Model. With a model of the vehicle defined, we formulate the trajectory planning problem for an autonomous racing car, and we describe several methods of finding a feasible plan in the chapter Trajectory Planning Problem. In the chapter Trajectory Following we will describe algorithms for following a predefined reference trajectory.

Finally, we implement the racing agent and we conduct several experiments to verify the performance and success rate of our algorithms on a physical robot in the chapter Experimental Verification, and summarize our findings.
