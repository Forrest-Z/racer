\chapter{Trajectory Planning}

The main focus of this thesis is the problem of trajectory planning for a fast moving car. In this chapter, we will analyze this problem in depth. We will first look into the planning problem in general and then we will discuss what the term ``fast moving cars'' means and how trajectory planning for a fast car is different from the general planing problem.

With the knowledge of the theory, we will formulate our trajectory planning problem. We will consider several well-known search algorithms used to solve planning problems and adapt them to our problem.

The solution of the planning problem will be a trajectory. A trajectory is the description of both a path the vehicle should follow and also a speed profile. If a robot is able to follow this trajectory, it will have an advantage over reactive steering algorithms, such as end-to-end driving, as it will be able to go through a series of corners efficiently by slowing down and accelerating at convenient times and by following an appropriate racing line. We will discuss a way of following the trajectory later in Chapter~\ref{chapter:following}.

\section{Introduction to Planning}

In this section, we will give a brief overview of the basic concepts of automatic planning and planning under differential constraints. The main source of the information for this chapter is a book by Steven M. LaValle called \textit{Planning Algorithms} \cite{lavalle_2006}. This book describes all of the topics in this section in much more detail and it is a good source of further information on this topic.

\subsection{Automatic Planning}

Planning is the task of solving some problem by finding a sequence of actions, which transforms the world to some desired state. The world of a planning problem is defined in terms of states and actions.

A single state is a full description of all of the important aspects of the world. A state can be encoded in many different ways, for example it can be a set of logical propositions which hold true in the given state or a vector of an $n$-dimensional vector of real numbers. All of the possible states of the world form a state space.

An action is a way of changing a state of the world into a different state. Not all actions can be applied in all states and so the set of possible actions can differ from state to state. All of the actions together then form an action space.

The planning task is to find a sequence of actions, which changes the state of the world from a given initial state to some desired state. Such sequence of actions is referred to as a feasible plan.

With the intuition of what a planning problem is, we can formulate it formally:

\begin{defn}[Planning Problem]
	\label{def:basic_planning_problem}
	A planning problem is a tuple $\left(X, U, f, x_0, g\right)$ consisting of:
	
	\begin{enumerate}
		\item A \textit{nonempty set} $X$ of world states, called the \textit{state space}.
		\item For each state $x\in X$, a set of actions $U(x)$.  A set of all actions $U=\bigcup\limits_{x\in X} U(x)$ is called an \textit{action space}.
		\item A \textit{state transition function} $f$ defined for every $x\in X$ and $u\in U(x)$, which produces the state of the world after applying the action $u$ to the state $x$.
		\item An \textit{initial state} of the world $x_0\in X$.
		\item A goal region $X_g\subset X$.
	\end{enumerate}
\end{defn}

\begin{defn}[Feasible Plan]
	A solution to a planning problem $\left(X, U, f, x_0, g\right)$ is a \textit{feasible plan}, which a finite sequence of actions $\langle u_0, u_1, \ldots, u_k\rangle\in U^*$ such that:	
	\begin{gather*}
		\forall i \in \left\{0,1,\ldots,k\right\}: u_i\in U(x_i) \wedge x_{i+1}=f(x_i, u_i) \\
		x_{k+1} \in X_g.
	\end{gather*}
\end{defn}

\begin{defn}[Optimal Plan]
	Let $\left(X, U, f, x_0, g\right)$ be a planning problem and $\Pi=\left\{\pi\in U^* \mid \pi \text{ is a feasible plan}\right\}$ a set of all feasible plans. We say that a plan $\pi^*\in \Pi$ is an optimal plan with respect to a cost function $\gamma: \Pi \rightarrow \mathbb{R}$ if
	
	\[
		\pi^*=\argmin_{\pi\in\Pi} \gamma(\pi).
	\]
\end{defn}

\subsubsection{Reachability Graph}

We can imagine that the states form vertices of a graph $G=(V, E)$ and the applications of actions through the state transition function form directed edges between the vertices:

\begin{equation*}
\begin{aligned}
	V&=X \\
	E&=\left\{(x_1, x_2) \mid \exists u \in U(x_1): x_2 = f(x_1, u) \right\}.
\end{aligned}
\end{equation*}

This graph can in general have several subgraphs. We are only interested in the component, which contains the initial $x_0$ and all the vertices which are reachable from $x_0$ via a directed path. A feasible plan is then a directed path in the graph starting in the initial state vertex $x_0$ and ending in any of the goal states vertices. Finding a path in a graph is well-studied problem and there are several efficient algorithms to solve it, such as the Dijkstra algorithm or its extension called A*.

The size of the state space and the action spaces has great impact on the way how we approach the planning problem and how we find a solution. If the graph is finite or if the the number of vertices is countably infinite and the branching factor is finite, we can find the solution (if one exists) with a systematic search algorithm in a finite amount of time. If no solution exists and the graph is infinite, the algorithm will be trying different plans infinitely. In practice this can be avoided with a limiting criterion, such as maximum length of a plan.

When the number of vertices is uncountably infinite or the branching factor is infinite, the problem becomes much harder and we cannot rely on a simple graph search anymore. We will soon see, that the state space of all of the vehicle configurations in our problem is uncountably infinite. These problems can be solved using \textit{sampling algorithms}.

We must keep in mind that the number of states of the system can be very high even if it is finite because the state space represents all of the combinations of the state variables of the world.

\subsection{Planning Under Differential Constraints}
\label{sec:planning_under_differential_constraints}

When describing the motion of a car-like robot on a two-dimensional plane, we assume that we can treat it as a rigid body, usually represented as a bounding rectangle. The configuration of the rigid body is then described by the pose of the vehicle: the $\left(x,y\right)$ Cartesian coordinates of a fixed reference point of the body in the world reference frame, and the heading angle $\theta$ between the longitudinal axis of the vehicle and the $x$ axis of the world reference frame. The $(x, y)$ location is in some bounded area $P\subset\mathbb{R}^2$ and the heading angle is an arbitrary angle $\theta\in\left[0,2\pi\right)$. This simple configuration space already has three continuous dimensions and it is uncountably infinite.

The kinematics and dynamics of a robots are typically described by differential equations. These equations give us the velocities at which the state of the robot changes when an action is applied. As an example of these constraints, we can look at a model of a \textit{simple car} \cite[Section~13.1.2.1]{lavalle_2006}.

\paragraph{Example}
The state space of a simple car will consist of the poses in an infinite 2D plane $(x, y, \theta)$ as we described it earlier. The control inputs are two dimensional vectors $\left(u_s, u_\varphi\right)$, where $u_s$ is the commanded speed of the vehicle in the direction perpendicular to the rear axis, and $u_\varphi$ is the steering angle of the front wheels. For a better understanding of this example, see the Figure~\ref{fig:simple_car}.

For a car with a wheelbase length $L\in\mathbb{R}$, its velocity can be approximated by this set of equations:

\begin{equation}
\begin{aligned}
	\dot{x}&=u_s \cos \theta \\
	\dot{y}&=u_s \sin \theta \\
	\dot{\theta}&=\dfrac{u_s}{L} \tan u_\varphi.
\end{aligned}
\end{equation}

\begin{figure}
	\centering
	\label{fig:simple_car}
	
	\begin{tikzpicture}[
		axis/.style={thin, densely dashed, gray}
	]
	
	% the axes
	\path[name path=AX] (-1,0)--(8,0);
	\draw[thick, ->] (-1,0)--(8,0) node[right]{$x$};
	\draw[thick, ->] (0,-1)--(0,6.5) node[above]{$y$};
	
	\def\stateTheta{45} % heading angle of the vehicle
	\def\statePhi{-24} % steering angle of the front wheels
	\def\L{2.5} % wheelbase of the vehicle
	\def\W{1.45} % axle width
	\def\w{2} % width of the vehicle
	\def\l{4.5} % length of the vehicle
	\def\wW{0.2} % wheel width
	\def\wL{0.5} % wheel length
	\def\xoffset{3cm}
	\def\yoffset{0.8cm}
	
	% the rotated vehicle
	\begin{scope}[xshift=\xoffset, yshift=\yoffset]
	\begin{scope}[rotate=\stateTheta]
    
   		% longitudinal axis
   		\path[name path=AL] (-2.5, \w/2) -- (\l+2, \w/2);
   		\draw[axis] (-2.5, \w/2) -- (\l+2, \w/2);

		% visualize theta
		\path [name intersections={of=AL and AX, by={X}}];
		\draw[rotate=-\stateTheta] (X)++(0.7,0) arc (0:\stateTheta:0.7) node[right, yshift=-1mm, xshift=1mm] {$\theta$}; % theta
    
    	% shape of the vehicle
		\draw (0, 0) rectangle (\l, \w);		
		
		% axle coordinates
		\def\rearX{\l/2 - \L/2}
		\def\frontX{\l/2 + \L/2}
		\def\rightY{\w/2 - \W/2}
		\def\leftY{\w/2 + \W/2}
		
		\coordinate (front) at (\frontX, \w/2);
		\coordinate (rearLeft) at (\rearX, \w/2 + \W/2);
		\coordinate (frontLeft) at (\frontX, \w/2 + \W/2);
		
		% visualize the wheelbase length		
		\draw[axis] (rearLeft) -- ++(0, 1);
		\draw[axis] (frontLeft) -- ++(0, 1);		
		\draw[<->, >=latex] ($ (rearLeft) + (0, 0.8) $) -- node[above, xshift=-1.5mm, yshift=-1mm] {$L$} ($ (frontLeft) + (0, 0.8) $);
		
		\draw[thick] (\rearX - \wL/2, \rightY - \wW/2) rectangle (\rearX + \wL/2, \rightY + \wW/2); % rear right wheel/tire
		\draw[thick] (\rearX - \wL/2, \leftY - \wW/2) rectangle (\rearX + \wL/2, \leftY + \wW/2); % rear left wheel/tire
		\filldraw[thick] (\rearX, \rightY + \wW/2) -- (\rearX, \w/2) circle (2pt) node[right, xshift=1mm] {$(x, y)$} -- (\rearX, \leftY - \wW/2); % the axle 
		
		% front axle
		\draw[thick, rotate around={\statePhi:(\frontX, \rightY)}] (\frontX - \wL/2, \rightY - \wW/2) rectangle (\frontX + \wL/2, \rightY + \wW/2); % rear right wheel/tire
		\draw[thick, rotate around={\statePhi:(\frontX, \leftY)}] (\frontX - \wL/2, \leftY - \wW/2) rectangle (\frontX + \wL/2, \leftY + \wW/2); % rear left wheel/tire
		\filldraw[thick] (\frontX, \rightY + \wW/2) -- (\frontX, \leftY - \wW/2); % the axle 
		
		% connect the axles
		\draw[thick] (front) -- (\rearX, \w/2);
		
		% visualize phi
		\draw[axis, rotate around={\statePhi:(front)}] (front) -- (\frontX + 2, \w/2);
		\draw (front)++(0.7,0) arc (0:\statePhi:0.7) node[right, yshift=-2mm] {$\varphi$}; % theta
		
	
    \end{scope}
	\end{scope}
	
	\end{tikzpicture}
		
	\caption{The simple car from the example has three state variables. The speed and the steering angle are action variables and they can change at any moment, so the vehicle can stop on a spot and change the steering angle instantaneously, which is of course not possible in a real world car.}
\end{figure}

The configuration space of all possible transformations of the robot and possibly additional variables required to keep the state of the kinematics and dynamics of the robot together form a continuous \textit{state space} $X$. In the example of the simple car, the state space would be just the configuration space itself, therefore $X=\mathbb{R}^2\times\left[0,2\pi\right)$, but different models could have more dimensions as we will see in Section~\ref{sec:vehicle_model}.

\paragraph{Obstacles}

Additionally, we must be able split the state space $X$ into two complementary subsets: the obstacle region $X_{obs}\subseteq X$ and the free region $X_{free}=X\setminus X_{obs}$. Only the states in $X_{free}$ can be entered safely while the states in $X_{obs}$ must be avoided to prevent collisions of the robot with obstacles.

\paragraph{State Transition Function}

In order to be able formulate the planning problem for a system with differential constraints, we must change the meaning of the \textit{state transition function} $f$ from the previous Definition~\ref{def:basic_planning_problem}. This function used to produce the next state $x'$ after an action $u$ is applied to a state $x$, i.e. $x'=f(x, u)$. When the state space is continuous, the outcome of an execution of an action depends on for how long it is being applied. Instead of a direct transition to the next state, the function $f$ will now express the velocity in the state space as defined by the differential constraints:

\[
	\dot{x}=f(x, u).
\]

The function $f$ must be defined for every $x\in X$ and $u\in U(x)$. The task of a planning algorithm is then to find an \textit{action trajectory} which produces a collision-free \textit{state trajectory} which reaches some goal state. An \textit{action trajectory} is a continuous function $\tilde{u}: T \rightarrow U\cup\left\{u_t\right\}$ which maps an infinite interval $T=\left[0, \infty\right)$ to an action, which should be executed at the given point in time. From some point in time $t_{end}\in\left[0, \infty\right)$, a termination action $u_t$ can be applied to mark the end of the action trajectory ($\forall t>t_{end}: \tilde{u}(t)=u_t$). To \textit{state trajectory} corresponding to the action trajectory $\tilde{u}$ is a function $\tilde{x}: T \rightarrow X$, such that $\tilde{x}(0)$ is a given initial state $x_0$, and $\forall t \in \left(0, t_{end}\right]$:

\begin{equation}
	\label{eq:integrate_state_trajectory}
	\tilde{x}(t) = \tilde{x}(0) + \int_0^t f(\tilde{x}(\tau), \tilde{u}(\tau)) d\tau,
\end{equation}

such that $\forall \tau\in\left[0, t_{end}\right]: \tilde{u}(\tau)\in U(\tilde{x}(\tau))$. A collision-free trajectory adds a requirement to go only through the free region ($\tilde{x}(\tau)\in X_{free}$).

\begin{defn}[Planning Problem Under Differential Constraints]
	\label{def:planning_problem_under_differential_constraints}
	A planning problem under differential constraints $\left(X, U, f, x_0, g\right)$ is an extension planning problem given in Definition~\ref{def:basic_planning_problem}, such that:
	
	\begin{enumerate}
		\item The state space $X$ is continuous.
		\item A \textit{state transition function} $f$ defined for every $x\in X$ and $u\in U(x)$, which produces the velocity in the state space after applying the action $u$ to the state $x$: $\dot{x}=f(x, u)$.
	\end{enumerate}
\end{defn}

\begin{defn}[Feasible Plan]
	A solution to a planning problem under differential constraints $\left(X, U, f, x_0, g\right)$ is a \textit{feasible action trajectory} $\tilde{u}$ when $\exists t_{end} \in \left(0, \infty\right)$ such that:
	\begin{gather*}
	\forall t \leq t_{end}: \tilde{u}(t)\in U(\tilde{x}(t)) \\
	\tilde{x}(t) = \tilde{x}(0) + \int_0^t f(\tilde{x}(\tau), \tilde{u}(\tau)) d\tau \\
	\tilde{x}(t_{end})\in X_g.
	\end{gather*}
\end{defn}

\subsubsection{Sampling-Based Planning Methods}

In order to be able to plan in a world with a continuous state space and continuous time, sampling-based methods can be used. The main idea is to discretize the action trajectory by using the same action for some non-trivial time period. This effectively discretizes time into time intervals, which usually have a fixed length $\Delta t$ or the duration depends depend on the action which is used (``a motion primitive'' \cite[Section~14.2.3]{lavalle_2006}). The action trajectory can then be expressed as a simple sequence of actions and for every action in the sequence it is possible to determine the time at which it is supposed to be applied and for how long it is supposed to be applied. We can still obtain a continuous trajectory through the state space by integrating the action sequence.

Ideally, every segment of a state trajectory obtained by integrating some action from some state over some time period should be checked for collision. In practice, this could be very resource intensive, and so only a few samples (sometimes only the end state of the segment) are tested using a ``black box'' collision detection function to see if the segment is in $X_{obs}$ or $X_{free}$.

If the action space is infinite, it should be sampled as well and only a finite number of actions should be considered. This makes it possible to construct a graph where samples of $X$ are the vertices and the edges between them are actions. Starting from $x_0$, we compute all of the possible state trajectories starting in this state with all the actions in the finite set of actions over a time period (either a fixed $\Delta t$ or a time associated with the action) and add the final states as vertices to the graph and connect them with directed edges to the origin state. We repeat this for all added states over and over again.

The size of the graph is affected also by the discretization of time. The shorter the time interval is, the larger the graph will be. Nevertheless, we can perform graph search over and find solutions to planning problems with differential constraints. We will describe concrete search algorithms for our problem in Section~\ref{sec:trajectory_planning_algorithms}.

The discretization of time might make some states of the state space unreachable and this could make finding a feasible plan impossible. In order to achieve resolution-completeness of the sampling based algorithm, every time a plan is not found, the discretization should be made finer (e.g., by decreasing the length of the time intervals by half) \cite[Chapter~14.2]{lavalle_2006}.

\section{Trajectory Planning For Fast Moving Cars}

In this section we will discuss what it means to plan a trajectory for a fast moving car. What is the difference between trajectory planning for a slow moving car and for a fast moving car? What do we even consider to be a ``fast moving car'' and what is just a slow moving car?

A fast moving car is not a widely used term and it does not have a formal definition. Intuitively, we would consider a ``fast moving car'' to be a Formula 1, a rally car, or a different racing car. The racing drivers push their vehicles to their limits in order to be the first one behind the finish line. We might also consider ordinary driving on a motorway to be ``fast'', especially during a lane change maneuver or when we accelerate to overtake a vehicle in front of us. We might even consider driving at lower speeds to be ``fast'' when it involves taking sharp turns.

One thing these scenarios have in common is that the car reaches a speed at which it can become hard to steer the car in a specific direction because the tires might lose grip and the car might start moving sideways or in a more extreme scenario the car could roll over sideways. The speed which could be considered ``fast'' varies based on the radius of the turn.

For the purposes of this thesis, we will formulate the term in the following definitions:

\begin{defn}[Handling limits of a vehicle]
	We say, that a car is moving at its handling limits during a turn of a radius of $r\in(r_{min}, \infty]$ meters, if it is driving close to a speed $v_{max,r}$ at which the total force vector acting on the body of the vehicle would exceed the forces of the tires which keep the vehicle traveling at the constant radius $r$.
\end{defn}

The minimum turning radius $r_{min}\in\mathbb{R}$ is a constant specific for a given vehicle. It refers to the minimum turning radius which the vehicle can achieve at a very low speed and with the maximum steering angle possible. We can consider driving straight as driving along a circle with the radius of $\infty$ meters.

\begin{defn}[Fast moving car]\label{def:fast_moving_car}
	A car is moving fast when it is approaching its handling limits during a turn.
\end{defn}

\begin{defn}[Trajectory planning problem for a fast moving car]
	We say that a trajectory planning problem for a fast moving car is a planning problem under differential constraints, whose solution is a time-optimal state trajectory and the vehicle does not at any point in time exceed its handling limits. The \textit{state trajectory} is expected to describe both the poses of the vehicle, as well as a speed profile.
\end{defn}


To achieve a time-optimal trajectory, we will search for a trade off between the length of the path and the speed at any given moment. To keep within the handling limits of the vehicle, our differential constraints must closely model the behavior of the vehicle. Even if our model of the vehicle describes its behavior well, it is more than likely that the robot will deviate from the planned trajectory at some point. The noise in sensor readings, imperfections in the actuators, or imprecise map of the track, all of these factors will contribute to errors in trajectory tracking. At some point, the difference between the planned path and velocity profile of the vehicle will be so large, that it will be advantageous to create a new plan from the current pose and speed of the vehicle. To make this possible, we need to be able to re-plan the trajectory in a short period of time.

Given our assumption, that we will most likely have to re-plan the trajectory at some point in the future anyway, it does not make too much sense to plan the trajectory for the whole circuit at once. The longer the trajectory we are planning, the longer it will take to find a suitable plan. Instead, we could split the track into multiple shorter segments and plan a trajectory only for a few of the segments directly in front of the current pose of the vehicle.

When the vehicle is moving fast, a late or too early decision to change the direction or adjust the speed of the vehicle might result in a crash or a sub-optimal trajectory. Fast reaction times are therefore key. If we plan a trajectory for only a limited stretch in front of the vehicle, we must be able to find the plan for the following stretch before we reach the end of the current reference trajectory. Failing to do this, the vehicle will not have any trajectory to follow and it would probably have to preform some kind of an emergency braking in order to prevent collisions, or it would be forced to resort to some simpler reactive steering strategy, which does not involve planning ahead and which would most likely be sub-optimal.

We can now summarize the discussion in this section into a set of requirements for the trajectory planning algorithm which will make it more suitable for fast moving cars:

\begin{itemize}
	\item The whole track should be split into smaller segments, so that we do not have to waste resources to plan a trajectory for segments which will most likely not be reached by the time we will need to re-plan the trajectory.
	\item The differential constraints of the vehicle must accurately describe capture the movement of the vehicle at its handling limits and we must be able to eliminate dangerous maneuvers, which could lead to a crash or a trajectory or which could not be followed closely by the vehicle.
	\item The search algorithm we will use must find good solutions which are time-optimal or close to time-optimal in a very short period of time.
\end{itemize}

\subsection{Problem Formulation}

\subsubsection{The Configuration Space and the State Space}

We only consider the motion of our vehicle on a two-dimensional flat surface. For simplicity, we will consider the shape of the vehicle to be a rectangle of a constant width and height, which fully encloses all of the parts of the body of the vehicle. This gives us a simple way of checking collisions with obstacles and by adjusting the size of the rectangle we can also add a small safety margin around the vehicle to stay on the ``safe side'' when closely passing obstacles.

Any position of the boundary rectangle in the 2D plane can be expressed as a rigid body transformation. We will define a \textit{configuration space} $C$ will consist of all these possible transformations of the vehicle. Each transformation can be expressed as a vector $\left( x, y, \theta\right)\in \mathbb{R}^2\times \left[0, 2\pi\right)$, where $\left( x, y\right)$ are the coordinates of the center of gravity of the vehicle (corresponding to the center of the rectangle) and $\theta$ is the angle between the $x$ axis of the coordinate system and the longitudinal axis of the vehicle, i.e. the heading angle of the vehicle.  This configuration space is continuous.

The complete \textit{state space} $X$ of the vehicle will be a combination of the configuration space $C$ and additional variables representing the kinematics and dynamics required by the differential constraints. We will define a function $\kappa: X\rightarrow C$ which returns the configuration of the vehicle in the given state. We will discuss the choice of appropriate vehicle models in Section~\ref{sec:vehicle_model} and we will define a concrete state space for each of the vehicle models.

\paragraph{Initial State}
At the start of the race, the vehicle is expected to be situated at a predefined starting location of the racing circuit with its heading angle identical to the direction of the race. The vehicle starts from a standstill with the wheels not spinning and the front wheels pointing the heading direction of the car.

During the race, the speed and the steering angle have to be measured using sensors. We must take in mind that this state is most likely slightly inaccurate due to the errors in the measurements and due to a delay between the measurement and the start of planning. By the time the planning algorithm finds a trajectory, the state vehicle will have already changed and it will be different from the initial conditions of the planning algorithm.

\subsubsection{Collision Detection}

A collision detection function $c_{R}: C \rightarrow \left\{T, F\right\}$ divides the configuration space into two parts $C_{free}$ and $C_{obs}$:

\begin{equation*}
\begin{aligned}
	C_{obs} &= \left\{x\in C \mid c_{R}(x)=T\right\} \\
	C_{free} &= C \setminus C_{obs}.
\end{aligned}
\end{equation*}

The subscript $R$ represents a specific instance of the collision detection function for a rectangular shape of a specific width and height. The task of the collision detection function is to determine, if the boundary rectangle of the vehicle in the given configuration $x\in C$ collides with some obstacle in the world or not. This function is the only way how the planning algorithm can determine the shape of the track and any additional obstacles which appear on the track.

The planning algorithm does not work directly with the configuration space, but rather with a state space. An extension of the partitioning into $C_{free}$ and $C_{obs}$ into the state space is very straightforward because a collision is dependent only on the configuration of the vehicle body, not on the kinematics and dynamics of the vehicle. For a state $x\in X$, which is an extension a configuration $c_x \in C$, it holds $x\in X_{obs} \iff c_x\in C_{obs} \wedge x\in X_{free} \iff c_x\in C_{free}$.

\paragraph{Representation of Obstacles}
We represent the obstacles in the world as an occupancy grid. An \textit{occupancy grid} is a matrix $O_r^{m\times n}\in \left\{0, 1\right\}^{m\times n}$, where $m,n\in\mathbb{N}$ is the size of the grid and $r\in\mathbb{R}$ is the resolution. Every element of the matrix represents a square tile of $r\times r$ meters placed in a grid covering an area of $rm\times rn$ square meters. The tiles can be in two states: either it can be traversed freely, or there is some obstacle in the tile and so the tile must be avoided. This is represented by the values $0$ (free space) and $1$ (an obstacle) in our definition.

To check if a single point of the world $\left(x,y\right)\in \mathbb{R}^2$ collides with an obstacle or not, we must first determine the coordinates of the tile in which it belongs:

\begin{equation*}
\begin{aligned}
	x_r &= \left\lceil \frac{x}{r} \right\rceil \\
	y_r &= \left\lceil \frac{y}{r} \right\rceil
\end{aligned}
\end{equation*}

If the point lies outside of the area covered by the occupancy grid, we can treat it as an obstacle. Otherwise we look into the appropriate cell of the matrix. The collision detection function $c_1$ for a single point can be:

\begin{equation*}
	c_1(O_{r}^{m\times n}, x, y) =
	\begin{cases}
		1 & x_r < 1 \lor y_r < 1 \lor y_r > m \lor x_r > n \\
		\left(O_{r}^{m\times n}\right)_{y_r, x_r} & \text{otherwise}.
	\end{cases}
\end{equation*}

We assume that the occupancy grid is aligned with the origin of the world reference frame and that the column indexes grow parallel to the $x$ axis and the indexes of rows parallel to the $y$ axis. If there is a transformation between the origin of the world reference frame and the occupancy grid, the point $(x, y)$ has to be rotated and translated appropriately before the coordinates of the corresponding tile can be determined. Please note that matrix rows and columns are indexed from 1, unlike for example arrays in the C and many other programming languages, which start indexing at 0.

We chose the occupancy grid because it is a standard way of representing 2D maps in \gls{ROS}, especially when a robot is equipped with a \gls{LIDAR}. We used these technologies to build our custom experimental vehicle and therefore we also used this map representation.

\paragraph{Moving Obstacles}
In this thesis, we will not consider moving obstacles. If it was necessary, we would have to extend the definition of the collision detection function to include a time parameter and it would have to extrapolate and predict a state of the world at the given time and check the collision against this prediction. The rest of the planning algorithm would not be affected by this extension.

\paragraph{Collision Detection Implementation}
The collision detection function must be evaluated every time the planning algorithm considers a use of an action from some state. If the action would lead to a state which would collide with an obstacle, this action should not even be allowed. The complexity of the collision detection function will therefore have a great impact on the performance of the algorithm on real hardware.

The footprint of the body of the vehicle is a rotated rectangle and we have to check if any of the parts of this rectangle does not hit an obstacle cell. We considered two versions of the collision detection algorithm:
\begin{enumerate}
	\item Pre-calculating small occupancy grids for several configurations of the vehicle and overlaying them with the occupancy grid representing the world.
	\item Inflating the obstacles in the occupancy grid and performing a single point check.
\end{enumerate}

The idea of \textit{the first method} is to calculate a list of tiles relative to the tile in which the center of the rectangle lies which the vehicle could collide with. We would split the heading angle dimension of the configuration into several regular intervals. For each of these intervals we would take the value in the middle and use it as a heading angle $\theta$. We would then rotate the bounding rectangle by the angle $\theta$ and mark which cells of a grid with the resolution $r$ it would overlay. We must take into consideration that the center of the rectangle can be at any point of the tile and different cells would be overlaid for example when the center coincides with the top left corner of the cell or the bottom right corner. Figure~\ref{fig:collision_detection_overlap} depicts this idea. To check collisions, we would first determine the cell corresponding to the center of the bounding rectangle of the vehicle and determine the discretization f the heading angle $\theta$. For the discretized heading angle, we would then remember a list of integer coordinates of the cells which would be occupied relative to the center of the rectangle. We would then check the state of the occupancy grid for the tiles occupied by the footprint of the vehicle and if any of them was marked as $X$ or if the coordinate would be outside of the bounds of the $m\times n$ grid, we would report a collision.

\begin{figure}[!tbp]%
	\centering
	\begin{subfigure}[t]{0.45\textwidth}
		\missingfigure{FOOTPRINT}
		\label{fig:collision_detection_overlap}
		\caption{The relative coordinates to tiles occupied by the footprint of the vehicle when the heading angle is close to $\theta$.}
	\end{subfigure}
	\hfill
	\begin{subfigure}[t]{0.45\textwidth}
		\missingfigure{INFLATED OBSTACLES}
		\label{fig:collision_detection_inflation}
		\caption{The obstacles are ``inflated'' by the radius of the vehicle.}
	\end{subfigure}
	
	\caption{Photos of the experimental vehicle.}
\end{figure}

\textit{The second idea} is simpler. First, we determine the minimum safe distance of the vehicle. We will then change all free tiles in the occupancy grid to obstacle tiles if they are closer to an original obstacle tile than the minimum radius and ``inflate'' the obstacles. To check for collisions of a vehicle at position $(x, y)$, only a single invocation of $c_1$ is required, which has very little performance requirements. This method is depicted in Figure~\ref{fig:collision_detection_inflation}.

The obvious advantage of the first method is its higher accuracy. The second method could be insufficient for tasks such as perpendicular parking between two cars, where the whole space between the two cars could be filled with the inflated obstacles. In principle, we could fix this by ``inflating'' the occupancy grid for different heading angles and mark the footprints of rotated rectangles directly into the occupancy grid. The problem we had with this approach is an increased cost of pre-processing before the planning algorithm can start. The footprints of the vehicle can be pre-calculated ahead of time and they do not take any extra computation at runtime.

The occupancy grid can in principle change between two executions of the planning algorithm (e.g., a new obstacle is discovered) and therefore we might have to calculate the inflation before any invocation of the planning algorithm for all of the heading angles. In the end, the second approach was sufficient for the racing task and we used the simple implementation. The resulting function is sometimes ``to cautious`` and it reports collisions even if there is still some gap between the actual vehicle and the obstacle, but that can be considered as an advantage for high speed maneuvers and it might make sense to increase the radius even more to force the trajectory to be further from the obstacles in practice.

\subsubsection{Action Space}

Our vehicle has two actuators which can be controlled:
\begin{itemize}
	\item Steering servo which we can move to a desired position achieving a specific steering angle between the maximum left position, center position, and maximum rotation to the right.
	
	\item Motor which can be set to a specific \gls{RPM} and a direction of rotation when there is no load. The \gls*{RPM} will differ based on the load of the vehicle and the motor will require more voltage to achieve some RPM when the load is increased.
\end{itemize}

The electric signal we send to the actuators is interpreted as a target value and the actuator adjusts its output to match the target value over a time period at a rate which we cannot control. We define the \textit{action space} for these actuators like this:

\[
	U=\left\{ \left( \delta_t,\tau_t\right) \mid \delta_t,\tau_t\in\left[-1, 1\right] \right\}
\]

$U$ is a set of tuples of a steering angle proportion $\delta_t$, and a throttle position $\tau_t$. These actions are an abstraction of the signals which would be to the hardware. Negative throttle position $\tau_t$ means that the motor should spin in reverse, while positive values should result in the motor spinning in the normal direction of travel. When the vehicle is moving in some direction and the action specifies a throttle position in the opposite direction, the vehicle might engage brakes, if it is equipped with any. Since these actions represent merely the target values of the state of the actuators, our actions do not have any preconditions for the state in which they can be executed.

The action space $U$ as we defined it is infinite. In practice, our actuators can be set only to a fixed number of values. The \gls*{PWM} signal which controls the servo motors can only encode a finite number of levels. For a specific vehicle hardware, we can use only a finite subset $U_f\subset U$, $\abs{U_f}\in\mathbb{N}$ of valid actions.

\subsubsection{Discrete-Time Model}

Earlier in Section~\ref{sec:planning_under_differential_constraints}, we described an action trajectory as a continuous function of time which specifies the action at any given moment. This concept is unrealistic for our use case because we want to apply it to real hardware. The speed of an electric motor controlled by a \gls{PWM} signal can be adjusted only once per a duty cycle period. This information alone gives us a good reason to treat time not as a continuous function, but as a sequence of evenly spaced samples of time points, in which the planning algorithm can choose the next action. The choice of the sampling period will be a trade off between maximum possible control over the hardware and the number of decisions the planning algorithm will make.

We will integrate state transition function to integrate the velocity in the state space over a constant time period $\Delta t$ with a constant action $u\in U_f(x(t))$ based on (\ref{eq:integrate_state_trajectory}) and we will call the resulting function $f_{\Delta t}$ the \textit{system simulator}:

\begin{equation}
	x(t+\Delta t)=f_{\Delta t}(x(t), u)=x(t) + \int_{0}^{\Delta t} f\left(x(t+\tau), u \right) d\tau,
\end{equation}

where $\forall \tau \in [t, t+\Delta t]: x(\tau)\in X \wedge u\in U_f(x(\tau))$. The state trajectory function $x(t)$ can be reduced into a sequence $\langle x_0, x_1, x_2, \ldots, x_k\rangle$ and the action trajectory to a sequence of $\langle u_0, u_1, u_2, \ldots, u_{k-1}\rangle$, where the $i$-th step of the sequence $x_i$ corresponds to $x(i\Delta t)$, therefore we can write

\[
	x_{i+1}=f_{\Delta t}(x_i, u_i).
\]

Executing some actions for a time period $\Delta t$ might result in a collision with an obstacle. To avoid this problem, we will limit the action set of a vehicle state by incorporating the collision detection function and the system simulator $f_{\Delta t}$ to allow only safe actions:

\[
U_f(x)=\left\{u\in U_f \mid c(f_{\Delta t}(x, u)) = F\right\}.
\]

It is worth noting that if the vehicle is in state when its action set is empty, the vehicle is in a state when a collision within $\Delta t$ is inevitable.

\subsubsection{Goal Region}

When we analyzed the problem earlier in this chapter, we decided to plan a trajectory for the next $l>0$ track segments ahead (each segment should ideally correspond to a corner of the track or to a straight stretch between two corners). These segments will be given in the form of a list of points at the end of each segment in the coordinate frame of the track $\hat{w}=\langle \vec{w_0}, \vec{w_1}, \ldots, \vec{w_l} \rangle$.

To ensure that our vehicle goes around the circuit in the correct direction and it follows the track correctly even in cases, when the track intersects itself and forms loops, the trajectory of the vehicle must pass these waypoints in a the given order. The concept of ``passing a sequence of waypoints'' is formulated by the following definitions:

\begin{defn}[Directly Visible Points]
	We say that two vectors $\vec{x},\vec{y}\in\mathbb{R}^2$ are \textit{directly visible points}, if there is no obstacle in an occupancy grid $O_r^{m\times n}$ on the shortest path between these two points. The set $D\subseteq \mathbb{R}^2\times\mathbb{R}^2$ captures this relation: 
	\[
		(\vec{x}, \vec{y}) \in D \iff \forall t \in \left[0, 1\right]: c_1(O_{r}^{m\times n}, \vec{x} + t(\vec{y}-\vec{x})) = 0.
	\]
\end{defn}

\begin{defn}[Waypoint]
	Waypoint goal region is a set
	\[
		G_r(\vec{w})=\left\{c\in C \mid \norm{\vec{p_c}-\vec{w}}<r \land (\vec{p_c}, \vec{w}) \in D \right\},
	\]
	where $p_c$ is the position vector of the vehicle configuration $c$ and $r\in \mathbb{R}$ is the radius of the goal region. $G_r(\vec{w})$ is a set of all configurations which are considered to \textit{pass the waypoint}.
\end{defn}

The notation $\norm{\cdot}$ represents the Euclidean norm in $\mathbb{R}^2$. The choice of the radius of the \textit{waypoint goal region} is not too important. The radius should be large enough to cover the whole width of the track, so that it does not force the vehicle to change the trajectory just to pass the artificially added waypoint. On the other hand, the waypoints should not overlap very much. The requirement of \textit{direct visibility} is there to prevent scenarios when the vehicle could pass a waypoint ``behind a wall'' if the radius is too large, as is shown in Figure~\ref{fig:waypoints}.

\begin{figure}
	\centering
	\label{fig:waypoints}
	\missingfigure{WAYPOINTS AND "BEHIND THE WALL"}
\end{figure}

To implement this goal condition, we must extend the state space $X$ with one extra discrete dimension. This dimension will denote how many waypoints the vehicle has passed since the initial state and we will define a function $\mu: X\rightarrow \mathbb{N}$ which returns this number. For the initial state $x_0$ the number of passed waypoints will always be 0. For a state $x\in X$ the value is defined based on the state from which it was reached $x'\in X$: 

\begin{equation*}
	\mu(x)=
	\begin{cases}
		\mu(x')+1 & \kappa(x)\in G_r(\vec{w_{\mu(x')+1}}) \\
		\mu(x') & otherwise.
	\end{cases}
\end{equation*}

The goal region $X_g$ will be a set of all states, which pass all $l$ waypoints:

\[
	X_g=\left\{x\in X \mid \mu(x)=l\right\}.
\]

\subsubsection{Problem Formulation}

In the previous sections, we described and defined the individual components of the trajectory planning problem. We can now formulate it formally:

\begin{defn}[Trajectory Planning Problem]

	A trajectory planning problem for a fast moving car is a tuple $\left(X, U_f, f_{\Delta t}, c, x_0, X_{\hat{w}}\right)$:

	\begin{itemize}
		\item A non-empty continuous set $X\subseteq\mathbb{R}^n$, which is the state space of the vehicle. Each state of the vehicle is represented by $n\in\mathbb{N}:n\geq4$ state variables and it contains the dimensions of the configuration of the vehicle in a two dimensional plane, a discrete dimension for the number of passed waypoints, and other state variables as defined by the vehicle model which is used.

		\item A non-empty finite set $U_f$ of actions, such that $\forall x\in X, \forall u\in U_f(x): c(f_{\Delta t}(x, u)) = F$, where $c$ is a collision detection function.

		\item A \textit{system simulator} $f_{\Delta t}:X\times U_f \rightarrow X$ for a fixed time period $\Delta t>0$.
		
		\item An initial state of the vehicle $x_0 \in X$.

		\item A goal region $X_g\subset X$ for some fixed sequence of waypoints $\hat{w}=\langle w_0, w_1, w_2,\ldots, w_l \rangle, l>0$.
	\end{itemize}
\end{defn}

\subsubsection{Time-Optimal Feasible Solution}

The vehicle starts in an initial state $x_0$. It is controlled through an input sequence of actions applied over time at a constant rate, because we chose a fixed time interval $\Delta t>0$ between application of any two consequent actions. We can calculate the expected evolution of the state of the vehicle by integrating the velocity from a state transition function. If we take snapshots of the state every $\Delta t$ interval, we will get a sequence of states which we call a state trajectory:

\begin{defn}[Feasible State Trajectory]
	We say that a sequence of states $\hat{x_k}=\langle x_0,x_1,x_2,…,x_k \rangle ,x_i\in X$ is a feasible state trajectory starting in the state $x_0$ when for a fixed time interval $\Delta t>0$:
	\[
	\forall i \in \left\{ 0,\ldots,k-1\right\} \exists u\in U_f(x_i): x_{i+1}=f_{\Delta t} (x_i,u).
	\]
\end{defn}

We say that the state trajectory is feasible to emphasize the fact that it is collision-free. Remember that for each state $x$ we do not allow any actions which would lead to a crash to be in $U_f(x)$.

This sequence contains all the information we need to reconstruct the full continuous trajectory by finding the corresponding sequence of actions. In practice, this is not very important to us though. As we mentioned earlier, we do not expect the robot to be able to follow any plan perfectly, so we cannot execute the actions one by one. It is important for us to know in which the state the robot should at any given moment. The solution to our problem is therefore not a sequence of actions, but a state trajectory:

\begin{defn}[Feasible Solution]
	For an instance of a trajectory planning problem for a fast moving car $\left(X, U_f, f_{\Delta t}, c, x_0, g_{\hat{w}}\right)$, a feasible state trajectory $\hat{x_k}=\langle x_0, x_1, \ldots, x_k \rangle$ is a time-optimal feasible solution of the problem if:
	\begin{itemize}
		\item the final state is in the goal region: $x_k \in X_g$,
		\item and for every other feasible state trajectory $\hat{y_l}$, which ends in the goal region, it takes longer or the same time to reach the goal (i.e. $l \geq k$).
	\end{itemize}
\end{defn}

\paragraph{State Trajectory Sub-sampling}

The choice of the sampling time interval $\Delta t$ greatly impacts the performance of finding solutions to the planning problem. If the optimal time to reach the goal is $t$ seconds, then a state trajectory for a sampling interval $\Delta t_1$ will have more elements than a state trajectory with a sampling rate of $\Delta t_2 > \Delta t_1$. The planning algorithm would therefore have to make more decisions for $\Delta t_1$ and it would take longer to find a solution. On the other hand, a longer sampling interval could find invalid trajectories, because due to a long distance between two poses of the vehicle, the vehicle could ``jump through walls'' as the collision detection algorithm, which tests only the samples from the state trajectory, would give false positives.

Instead, we can uniformly subdivide the time interval of length $\Delta t$ into $n\in\mathbb{N}$ smaller intervals of $\frac{\Delta t}{n}$ and apply the state transition function $n$ times for every action in the sequence. This technique will give us better estimate of the motion of the vehicle and reduce the inaccuracy of numerical integration but at the same time it will not increase the computational complexity of the planning algorithm because we do not increase the number of explored sequences of actions. A good pair of $\Delta t$ and $n$ has to be found experimentally to provide good precision while keeping good performance of the algorithm on real hardware.

\section{Track Segmentation}

A natural way of splitting the track into smaller segments is to find the corners of the track. We can then plan the trajectory for the very next turn in front of the vehicle and for one or two  consecutive ones, and imitate the behavior of a human racing driver, as we described it in Section~\ref{sec:racing_line}. To prevent long segments for long straight stretches, we can set a limit for the length of a segment and split long segments into multiple shorter ones.

Another benefit of splitting the track into smaller segments and planning the trajectory just for a fixed number of them is that the total length of the track does not affect the performance of the algorithm anymore. The length of the segments is limited and so the actual time it will take to calculate a trajectory for the next $n$ segments on real hardware should be similar for different parts of the track. We will have to test this hypothesis experimentally.\todo{Actually test this.}

In this section we will describe an algorithm we chose to find the corners of a racing circuit. We will use this algorithm once just before the start of the race, to split the circuit into a series of short segments. We expect to be given the definition of a racing circuit as an occupancy grid, initial pose of the vehicle with respect to the occupancy grid, and at least two more checkpoints which define the direction in which the vehicle must drive along the circuit. An occupancy grid can be formalized with the following definition:

\begin{defn}\label{def:occupancy_grid}
	Occupancy grid $G\in\{0, 1\}^{m\times n}$ of resolution $r\in\mathbb{R}$ is a two dimensional table of $m$ rows and $n$ columns which corresponds to a rectangular area of the environment of the width of $m * r$ meters and the length of $n * r$ meters. The cells of the table fill the area as square tiles of the side length of $r$ meters. The value of a cell $G_{ij}$ reflects on the state of its corresponding area:
	
	\[
	G_{ij} =
	\begin{cases}
	-1\text{,} &\quad\text{if } i < 0 \vee j < 0 \vee i \geq m \vee j \geq n\\
	0\text{,} &\quad\text{or if the corresponding tile contains an obstacle} \\
	1\text{,} &\quad\text{otherwise.}
	\end{cases}
	\]
\end{defn}

The goal of the track analysis algorithm is to find interesting points of the track which split the track into smaller segments corresponding to stretches between the corners of the track. In order to achieve this, we can make a simple observation and derive an algorithm which will produce good approximate solutions.

We can inflate imaginary rubber walls with the thickness of a given safety radius of the vehicle along the edges of the track and loosely lay an imaginary string through the whole circuit. We can then start tightening the string and eventually it will take a form of alternating straight segments and parts, where it touches the rubber wall at an inner edge of a corner of a track. The string represents the shortest path for the vehicle around the circuit. We can then remove the imaginary rubber walls and start walking from the initial position of the vehicle along the string. We will mark the furthest point which is directly visible from the place where we're standing. By directly visible we mean that it is possible to draw a line between the two points in the occupancy grid and it will not intersect a cell containing an obstacle between the two points. This is the first corner ahead of us. We will then walk to this point and repeat the process, until we can see the first point we marked again. This thought process is visualized in Figure~\ref{fig:thought_process} on different track layouts.

\begin{figure}
	\label{fig:thought_process}
	\missingfigure{The thought process of finding the corners of a circuit.}
	\caption{The thought process of finding the corners of a circuit.}
\end{figure}

We will implement this process with a three step algorithm which will identify the corners and points along long winding bends. The first step will be to find the shortest path through the grid which starts at the initial position of the vehicle and which goes through the checkpoints in the correct order and at any point it does not come closer to an obstacle than to a distance of the safety radius. The second step will simply traverse the path once and select a sub-sequence of the points such that for two consecutive points $A$ and $B$, $B$ was the last point of the points immediately following $A$ on the original track which are directly visible from $A$. In the third step, we will merge points, which are close together (e.g., in a 180° hairpin turn).3

The shortest path can be found in several different ways. The simplest approach would be to use a simple grid search on the occupancy grid. An interesting alternative is the Space Exploration algorithm described by Chao Chen \cite{SEHS} which uses a search algorithm to explore the grid using circles of variable radii which depend on the distance to the closest obstacle. The expansion of a circle is achieved by calculating $k\in\mathbb{N}$ points on the circumference of the expanded circle and calculating maximum possible a radius for the given point as a distance to the closest obstacle. We will add the child circle to the open set if its radius is larger than some minimum radius (i.e., we will avoid points too close to obstacles) and if the circle has not been closed yet. A circle will be considered closed if the center of the circle lies inside of an already closed circle. This allows us to avoid exploring some regions of the occupancy grid multiple times. To search the space efficiently, we will use the A* algorithm as the search method. The cost to come to a circle will equal to the distance traveled from the initial position to the circle and the estimate of the cost to go will be equal to the euclidean distance to the goal position. We will stop searching at the moment when we expand a circle which contains the goal position. The outline of the algorithm is described in Algorithm~\ref{alg:space_exploration} and a visualization of the algorithm is shown in Figure~\ref{fig:sehs_space_exploration}.

\vspace{1cm}
\begin{algorithm}[H]
	\SetAlgoLined
	\DontPrintSemicolon
	
	\SetKwFunction{Top}{Top}
	\SetKwFunction{MaxRadius}{MaxRadius}
	\SetKwFunction{PointsOnCircumference}{PointsOnCircumference}
	\SetKwFunction{ReconstructPath}{ReconstructPath}
	
	\KwIn{Occupancy grid $G$, starting position $\vec{x}_0$, goal position $\vec{g}$}
	\KwOut{Sequence of circles}
	\Parameter{Number of expanded children $k$, minimum radius $r_{min}$}
	
	$r_0\gets$ \MaxRadius{$G$, $\vec{x}_0$}\;
	$O\gets\{(\vec{x}_0, r_0)\}$ \Comment*[r]{Open set}
	$C\gets\emptyset$ \Comment*[r]{Closed set}
	$P\gets\emptyset$ \Comment*[r]{Set of transitions}
	
	\While{$O \neq \emptyset$}{
		$(\vec{x}, r)\gets $\Top{$O$}\;
		$O\gets O\setminus\{(\vec{x}, r)\}$\;
		
		\If{$\|\vec{x} - \vec{g}\| \leq r$}{
			\KwRet \ReconstructPath{$(\vec{x}, r)$, $P$}\;
		}
		
		\For{$\vec{p'}$ in \PointsOnCircumference{$(\vec{x}, r)$, $k$}}{
			$r'\gets$ \MaxRadius{$G$, $\vec{p'}$}\;
			
			\If{$r'\geq r_{min}\ \wedge\ \not\exists (\vec{x_{c}, r_{c}})\in C: \|\vec{x_c} - \vec{p'}\| \leq r_c)$}{
				$O\gets O\cup \{(\vec{p'}, r')\}$\;
				$P\gets P\cup \{\big((\vec{x}, r), (\vec{p'}, r')\big)\}$\;
			}
		}
		
		$C\gets C\cup\{(\vec{x}, r)\}$\;
	}
	
	\caption{Space Exploration}
	\label{alg:space_exploration}
\end{algorithm}
\vspace{1cm}

To find the path from the initial position through the check points and to the finish line position, we will simply find the path from the initial point to the first checkpoint and then starting from the last circle of the previous path to the next check point. We repeat this until we close the circuit by reaching the initial position again. The path of circles we find might not be optimal. We can further improve it by iterating over the path it and smoothing it as shown in algorithm~\ref{alg:sehs_smoothing}.

With the path of circles around the circuit, we can now proceed the second step of finding the corners of the circuit. We will traverse the path and mark the centers of circles which are the last directly visible points from the previous point as shown in algorithm~\ref{alg:find_apexes}

\vspace{1cm}
\begin{algorithm}[H]
	\caption{Waypoint Selection}
	\label{alg:find_apexes}
	
	\SetAlgoLined
	\DontPrintSemicolon
	
	\SetKwFunction{AreDirectlyVisible}{AreDirectlyVisible}
	
	\KwIn{Occupancy grid $G$, sequence of $n$ points $(\vec{x_i})_{i=0}^{n}$}
	\KwOut{Sequence of points}
\end{algorithm}
\vspace{1cm}

% todo algorithm smoothing out

% do I need to prove 

\section{Differential Constraints}
\label{sec:vehicle_model}

\section{Trajectory Planning Algorithms}
\label{sec:trajectory_planning_algorithms}

With a complete formulation of the trajectory planning problem for a fast moving car, we can design a planning algorithm which will find a time-optimal feasible solution trajectory starting at any given initial state and which will pass any sequence of waypoints it is given. Our planning algorithm must respect the differential constraints of our vehicle represented by the state transition function and avoid collisions with any obstacles by utilizing the collision detection function.

In this chapter we will introduce several well-known sampling-based algorithms and their variants which we considered for solving our trajectory planning problem. We will describe the main ideas behind these algorithms and discuss if and how they could be used in our specific use case.

\subsection{Rapidly-Exploring Random Trees}

The \gls{RRT} is a simple randomized algorithm designed by Steven M. LaValle with differential constraints of mobile robots in mind \cite{RRT}. The core idea of this algorithm is to build a tree structure rooted in the initial state $x_0$ with branches representing state trajectories by randomly sampling the state space and using the system simulator function to find feasible trajectories in the state space.

The tree is grown iteratively by randomly sampling a state $x_{rand}\in X$ using some sampling scheme and choosing an action $u$ from $U_f$ which will steer the robot to a new state $x_{new}$ towards $x_{rand}$ without colliding with an obstacle from an existing tree node $x_{near}$ which is the closest to $x_rand$ according to some metric $\rho$ on $X$. The new node is added as a vertex to the tree and it is connected via an edge to its parent node $x_{near}$. This process is repeated until a state trajectory which ends in the goal region is found or until a fixed number of iterations is exceeded, in which case the search ends with a failure. The pseudocode of RRT is included in Algorithm~\ref{alg:rrt}.

\begin{algorithm}
	\SetAlgoLined
	\DontPrintSemicolon
	
	\SetKwFunction{AddVertex}{AddVertex}
	\SetKwFunction{AddEdge}{AddEdge}
	\SetKwFunction{TrajectoryFromBranch}{TrajectoryFromBranch}
	\SetKwFunction{SelectRandomSample}{SelectRandomSample}	
	\SetKwFunction{FindNearestNode}{FindNearestNode}
	
	\KwIn{Trajectory planning problem $(X, U_f, c, f_{\Delta t}, x_0, X_g)$, a metric on the state space $\rho$}
	\KwOut{State trajectory or $null$}
	\Parameter{Maximum number of iterations $K\in\mathbb{N}$}
	
	$T\gets \left(\left\{x_0\right\}, \emptyset\right)$\;
	
	\For{$k=1\ldots K$}{
		$x_{rand}\gets$ \SelectRandomSample{$X$}\;
		$x_{near}\gets$ \FindNearestNode{$T$, $x_{rand}$, $\rho$}\;
		$u_{best}\gets\argmin_{u\in U_f(x_{near})} \rho(x_{rand}, f_{\Delta t}(x_{near}, u))$\;
		$x_{new}\gets f_{\Delta t}(x_{near}, u_{best})$\;
		\AddVertex{$T$, $x_{new}$}\;
		\AddEdge{$T$, $\left(x_{near}, x_{new}\right)$}\;
		\If{$x_{new}\in X_g$}{
			\KwRet{\TrajectoryFromBranch{$T$, $x_{new}$}}\;
		}
	}
	
	\KwRet{null}\;
	
	\caption{The RRT algorithm.}
	\label{alg:rrt}
\end{algorithm}

\subsubsection{Algorithm Analysis}

\paragraph{Nearest Neighbor Search}

The algorithm performs a search for the closest state to the sampled state among the nodes in the tree according to the metric $\rho$ in every iteration of the algorithm and this search can have considerable impact on the performance of the algorithm.

A naïve solution would be to go over all tree nodes and find a node with minimum distance to the sampled state, but this approach would have linear time complexity with respect to the number of nodes of the tree. A better solution is to use a space partitioning data structure which is designed for nearest neighbor search. A good choice is a balanced $k$-d tree \cite{kd-tree}, where $k$ is the dimension of the partitioned space, in our case the dimension of $X$.

\paragraph{Time Complexity}

The algorithm will run for at most $K$ iterations. During every iteration, a random state will be sampled, nearest node will be found, an action will be selected, and the new state will be inserted into the already explored states tree structure.

We can assume that selecting a random sample of the state space does not depend on the already explored parts of the state space and that this operation time complexity is constant.

The complexity of finding nearest neighbor and inserting a new node into an $k$-d tree with a fixed number of dimensions is logarithmic with respect to the number of nodes in the tree on average, in our case this means $O(\log K)$.

Finding an action which contributes to the largest progress towards the goal will be implemented by evaluating all the $m=|U_f|$ options and measuring the distance between the resulting state and the target state. The time complexity of this operation is $O(m)$.

Checking if a state is a goal state is a constant operation because we only have to check if $\mu{x_{new}}$ equals to the number of waypoints.

The time complexity of a single iteration of the algorithm is $O(\log K + m)$ and so the total time complexity of the algorithm is $O\left(K(\log K+m)\right)$.

\paragraph{Memory Complexity}

In order to capture the state of the algorithm, we need to keep the tree of all already explored states. As we discussed earlier, we use a $k$-d tree because of its nearest neighbor search queries. The memory complexity is linear with the number of nodes stored in the tree. In each iteration of the algorithm we add at most one new node to the tree which will lead to a tree with at most $K$ nodes. The memory complexity of the whole algorithm is therefore $O(K)$.

\paragraph{Completeness}

All nodes added to the tree are added as leaves and an edge between two nodes in the tree always corresponds to application of some action $u\in U_f$. None of the nodes added to the tree corresponds of a state which would collide with some obstacle.

All branches of the tree $T$ correspond to some feasible trajectory in the state space. The algorithm will stop as soon as the tree contains a branch which corresponds to a feasible solution trajectory or if the number of iterations exceeds the limit $K$.

\gls*{RRT} is proven to be probabilistically complete for $K$ approaching infinity \cite{RRT_star}. This means that if we sample the state space long enough and make our tree denser and denser, we are guaranteed to eventually find some solution. Because we chose the discrete-time model, we would have to also make the time discretization finer after each failure to achieve completeness.

\paragraph{Optimality}

The main difference between our modified algorithm and the original algorithm proposed by Steven LaValle is the early termination of the algorithm when the first solution is found \cite{RRT}. We do not try to find a better solution and we do not grow our tree in a way which would guarantee that the first solution we find will be an optimal solution.

\gls*{RRT} does not have any mechanism for finding optimal trajectories. On contrary, it is proven that the paths found by \gls*{RRT} are optimal with probability of 0 \cite{RRT_star}.

There are modifications to the \gls*{RRT} algorithm which utilize the randomized sampling approach and probabilistic completeness of the algorithm, but which are optimal. One of these algorithms is called \gls{RRT*}. Unfortunately, this algorithm is not easily usable for us, because it is not designed for problems with differential constraints. The algorithm uses a ``rewiring'' procedure to optimize the tree edges after a new node is added. This procedure needs to be able to connect two states $x_1, x_2\in X$ with an action trajectory. This would be near impossible in the discrete-time model we chose and even with variable time, this operation would be very expensive and impractical.

\subsubsection{Sampling Schema}

The sampling schema is the most important part of the algorithm when it comes to the rate of convergence to a solution. The sampling function can also have an impact on the completeness of the algorithm.

\paragraph{Uniform sampling} is the simplest form of random node selection. The value for each dimension of the state vector is selected uniformly from the set of valid values.

\paragraph{Goal biasing} is a technique where we choose a state representing the final configuration with probability $p$ and we uniformly sample the space otherwise. This method can be generalized for several goals which can then be useful to bias the exploration of the state space in the direction of several waypoints. The goal of this method is to drive the search towards the goal faster, but at the same time keep the probabilistic completeness of the algorithm, as uniform sampling can help avoid getting stuck in dead ends.

\paragraph{Path biasing} is an extension of the goal biasing sampling scheme. We first find a reference path using a relaxed variant of the problem (e.g., a grid search algorithm without considering the differential constraints) and choosing the points along the path as intermediate goals for the grown tree. This variant of the algorithm is referred to as \textit{RRT-Path} \cite{RRT_guiding_path}.


\subsection{Hybrid A*}

Due to the infinite number of states reachable from an initial state $x_0\in X$ through actions from the infinite set of actions $U$ and the transition function $f_{\Delta t}$, it is impossible to exhaustively search for an optimal trajectory in the configuration space. Even if we limit the number of actions to a finite set of motion primitives $U_f$ and limit the maximum length of a trajectory, uninformed search for a trajectory will be impractically slow. In order to converge to a goal faster, informed search algorithms estimate graph nodes based on knowledge of the nature of the searched space.

A* is a complete and optimal informed graph search algorithm \cite{Nilsson_astar}. It visits the most promising nodes of the graph first, but it will eventually visit all the reachable nodes, until it finds a path to a goal state, or until every node was visited. We will describe a modified version of the A* algorithm which reduces the searched state space even further by discretizing the original continuous state space into an $n$-dimensional grid while still producing feasible trajectories in the configuration space.

This modified algorithm is referred to as \textit{Hybrid-State A* Search}. It was successfully used for path planning by the Stanford Racing Team in their vehicle \textit{Junior} during the DARPA Urban Challenge in 2007 \cite{dolgov08gppSTAIR}. This algorithm was utilized for navigation in unstructured environments when it was not possible to follow road centerline such as parking lots and for other complex maneuvers such as U-turns or overtaking of other vehicles. It has been explored and successfully used by several other researchers since then \cite{Hybrid_astar}.

\subsubsection{State Space Discretization}

The advantage of A* is that it visits every state at most once and whenever we reach a graph node is visited, we have an assurance that the path through which we came from the initial node is the best possible one and we can ignore this node in the future if we find a different route to it. This works well for discrete search spaces where it is easy to compare two nodes between themselves and determine whether it was already visited or not. In a continuous state space, this can be problematic. Even though we come very close to an already visited node, it does not have to be equal to the node. Let us consider the following example.

\paragraph{Example} Imagine we have a robot which moves in a two-dimensional plane and it is subject to differential constraints. The state space will be the configuration space $X=C$. We will focus on one action $u$ which moves the robot forward and steers to the left. In a fixed time of $\Delta t=1$ second the robot should travel around a quarter of a circle and turns by ninety degrees ($\dfrac{\pi}{2}$ radians). If we start from an initial state $x_0=\left(0,0,0\right)$ then after applying $u$ four times in a row, we would expect to go through states $x_1=\left(1,1,\dfrac{\pi}{2}\right),x_2=\left(0,2,\pi\right),x_3=\left(-1,1,\dfrac{3\pi}{2}\right)$, and arrive to a final state $x_4=\left(0,0,0\right)=x_0$ after 4 seconds. If we try to implement and run this example in a simulation, we might get a slightly different result caused by accumulated rounding errors and other numerical approximations. For example, we might use only a finite number of the decimals of the number $\pi$ and arrive to a heading angle of $\theta<2\pi$. The A* algorithm will consider $x_0$ and $x_4$ as different nodes of the graph and it will continue with the expansion of $x_4$. This example is visualized in Figure~\ref{fig:no_discretization_example}.

\begin{figure}
	\centering
	\label{fig:no_discretization_example}
	\missingfigure{Misaligned rotation image.}
	\caption{We would expect the car to return to its initial position after the action $u$ was applied 4 times. Due to rounding errors and errors of numerical integration, we might reach a different state very close to the initial state.}
\end{figure}

\textit{Hybrid A*} divides the state space into a grid of similar states to avoid this problem. Before we expand a state from the configuration space, we will check if the grid cell it falls into is already closed or not. If the cell is closed, we will consider the state to be closed as well and we will not expand it again.

We can select an appropriate resolution for every dimension of the space base on the semantics of each dimension. If we wanted to ignore some dimension of the state space for searching, we can set the resolution for the dimension to $\infty$. We can convert the continuous dimensions of a state vector $x$ into a discrete grid coordinate with cell sizes vector $\sigma$ for the dimensions with this expression:

\[
	\tilde{x} \cong \left\lfloor \dfrac{x}{\sigma} \right\rfloor.
\]

We will refer to the discrete coordinates of the state in the state space by adding the ``\textasciitilde'' sign to the variable name.
The resulting state trajectory consisting of the discretized states might not be feasible. The \textit{hybrid A*} algorithm therefore remembers the continuous state through which we entered every visited discrete grid cell. We will continue the expansion by applying the state transition function to the original continuous state. The final trajectory will therefore be smooth and feasible because it will not be affected by the discretization as can be seen in Figure~\ref{fig:hybrid_astar_discretized_vs_continuous}.

\begin{figure}
	\centering
	\label{fig:hybrid_astar_discretized_vs_continuous}
	\missingfigure{Rough discretized and smooth paths.}
	\caption{}
\end{figure}

There is one problem with the discretization which is worth solving. Depending on the resolution of the grid, the next state obtained by the state transition function might fall into the same cell as the source state. This could be a significant problem especially at low speeds, for example when the vehicle is accelerating from a standstill. To solve this problem, we will repeat the same action $k$ times, until the resulting state falls into a different cell. We will then store the resulting state for future expansion with an appropriate cost.

\subsubsection{Cost-to-Come Function}

Our goal is to find the fastest trajectory possible. To achieve this, we must minimize the time it takes for our vehicle to get from the initial state to a goal state. We fixed the time $\Delta t$ we use for the system simulation function $f_{\Delta t}$. This means that the costs of our edges are uniform, and the cost is exactly the constant $\Delta t$. Sometimes we unfold one action $k$ times in order to leave the original grid cell and therefore the cost of this transition is proportionally higher. The cost function $c$ of a search node $v\in V$ can be defined recursively:

\begin{equation*}
\begin{aligned}
	g(x_0)&=0 \\
	g(x_{i+1})&=g(x_i)+k\Delta t,
\end{aligned}
\end{equation*}

where $x_0$  is the initial vehicle state, $x_i, x_{i+1}\in X, k\in\mathbb{N}$ such that $k$ is the number of unfolding of some action to reach $x_{i+1}$ from $x_i$. It would be possible to add additional costs for example for driving close to obstacles or for changing steering angle left and right too frequently, but we will use simply the time as the cost.

\subsubsection{Heuristic Function}

The A* algorithm is driven towards the goal by a heuristic function. Designing a good heuristic function can help us converge towards the goal faster by expanding the most promising nodes first. The heuristic function estimates the remaining cost to go from any state to the goal. Our cost function represents time it takes to drive from one state to another. Our heuristic function therefore must estimate the remaining time it will take to reach the goal from a given state.

To guarantee optimality of A* for graph searching, we require the heuristic to be admissible and monotonous. This means that the along any direct path from the initial node and a goal node the value of the heuristic must not increase and therefore maintain a form of triangle inequality.

The inspiration for our heuristic functions were taken from the Stanford Racing Team entry in the DARPA Urban Challenge \cite{}.

\paragraph{Euclidean Distance}
One of the simplest heuristics is calculating the distance between the position of the vehicle and the position of the goal. Our vehicle must drive past several waypoints. We must calculate the distance from the current position of the vehicle to the next waypoint and then add the distances between the remaining waypoints to obtain the minimum remaining distance $d$.

To calculate the time to reach the last waypoint, we must factor in the velocity of the vehicle. To maintain admissibility of the heuristic, we must assume that the vehicle will travel at maximum velocity. We will calculate the minimum time necessary by dividing the distance $d$ by the maximum possible velocity of the vehicle $v_{max}$.

This heuristic is admissible and monotonous, because as we move closer to the goal along any path, the Euclidean distance decreases and our estimate is always lower or equal to the real capabilities of our vehicle.

We ignore the driving characteristics of our vehicle and we also do not account for collisions with any of the obstacles or the boundary of the racing track. On the other hand, the value of the heuristic is easy to calculate, and it does not need any additional memory.

\paragraph{Shortest Path Through Waypoints}
We discretize the two-dimensional plane into a grid with the same resolution as we do for the position dimensions of the state space. We will then compute the distance to the goal in an eight-connected grid using dynamic programming. We ignore grid cells which are outside of the racing track or the ones which overlap with some obstacle. Given a distance, we can calculate the minimum time to reach the goal in the same way as in the case of Euclidean distance heuristic.

This heuristic ignores the capabilities of the vehicle and the constraints on steering. Nevertheless, it will guide the search along the course of the racing track, and it will help us avoid dead ends.

\paragraph{Combination of Heuristic Functions}
To take advantage of several different heuristics $h_1,h_2,\ldots ,h_n$, we can create a new heuristic, which takes the maximum estimated cost-to-go:
\[
	h(x)=\max_{i \in {1, 2, \ldots, n}}h_i(x).
\]

This heuristic will dominate all the original heuristics and it will give us better estimates of the remaining cost. We will the combination of both the euclidean distance and shortest path heuristics as our final heuristic for the A* algorithm.

\subsubsection{Total Cost Estimate}

We will combine the cost-to-come $g$ and the estimate of cost-to-go $h$ into an estimate of the total cost $f$ of the trajectory from the initial state to some goal state through any state $x\in X$:

\[
	\Sigma(x) = g(x)+h(x).
\]

In literature, the cost-to-come is sometimes referred to as $g$-value, the cost-to-go heuristic as $h$-value, and the estimate of the total cost the $f$-value. We already used the letter $f$ for the state transition function, so we will instead use the symbol $\Sigma$ to represent the sum of the two components of the estimate.


\begin{algorithm}[]
	\SetAlgoLined
	\DontPrintSemicolon
	
	\SetKwFunction{HybridAStar}{HybridAStar}
	\SetKwFunction{ReconstructTrajectory}{ReconstructTrajectory}
	\SetKwFunction{Expand}{Expand}
	\SetKwFunction{Open}{Open}	
	
	\KwIn{Trajectory planning problem, maximu cost limit}
	\KwOut{State trajectory or $null$}

	\vspace{3mm}

	\SetKwProg{astar}{Algorithm}{:}{}
	\astar{\HybridAStar{$(X, U_f, c, f_{\Delta t}, x_0, X_g)$, $g_{max}$}}{

	$O\gets\left\{x_0\right\}$\;
	$C\gets\emptyset$\;
	$E\gets\emptyset$\;
	
	\While{$O\neq\emptyset$}{
		$x\gets\argmin_{x'\in O} \Sigma(x')$\;
		$O\gets O\setminus\left\{x\right\}$\;
		$C\gets C\cup\left\{\tilde{x}\right\}$\;
		\If{$x\in X_g$}{
			\KwRet{\ReconstructTrajectory{$E$, $x$}}\;
		}
		\If{$g(x) < g_{max}$}{
			\Expand{$x$, $O$, $C$, $E$}\;
		}
	}
	
	\KwRet{null}\;
	}

	\vspace{3mm}
	
	\SetKwProg{procedure}{Procedure}{:}{}
	\setcounter{AlgoLine}{0}
	\procedure{\Expand{$x$, $O$, $C$, $E$}}{
		\ForEach{$u\in U_f$}{
			$x'\gets x$\;
			$k\gets 0$\;
			\Repeat{$x'=x\land \tilde{x}\neq\tilde{x'}$}{
				$x'\gets f_{\Delta t}(x', u)$\;
				$k\gets k+1$\;
			}
			\If{$\tilde{x'}\notin C\land c(x')=F$}{
				\eIf{$\exists x''\in O: \tilde{x'}=\tilde{x''}$}{
					\If{$g(x')<g(x'')\lor (g(x')=g(x'') \land \tau(x') > \tau(x''))$}{
						$O\gets O\setminus \left\{x''\right\}$\;
						\Open{$O$, $E$, $x$, $x'$, $k$}\;
					}
				}{
					\Open{$O$, $E$, $x$, $x'$, $k$}\;
				}			
			}
		}
	}

	\vspace{3mm}
		
	\setcounter{AlgoLine}{0}
	\procedure{\Open{$O$, $E$, $x$, $x'$, $k$}}{
		$O\gets O\cup\left\{x'\right\}$\;
		$E\gets E\cup \left\{\left(x, x'\right)\right\}$\;
		$g(x')\gets g(x)+k\Delta t$\;
	}
		
	\caption{The Hybrid A* algorithm.}
	\label{alg:hybrid_astar}
\end{algorithm}
