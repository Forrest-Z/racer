\chapter*{Conclusion}
\addcontentsline{toc}{chapter}{Conclusion}

In this thesis, we approached the problem of autonomous racing inspired by the F1/10 competition. This problem is a simplification of full-scale autonomous vehicle driving which allowed us to focus solely on collision-free time-optimal car-like vehicle motion without having to take into account other intricacies of the adjacent problems, such as obeying traffic rules and predicting the behavior of other road users and pedestrians.

Our approach to solve the problem was to create an agent, which analyzes the layout of the racing circuit and then plans its motion through the track focusing on just a fixed number of corners ahead of the vehicle. Using the Hybrid A* and the \gls{SEHS} algorithms, we were able to generate nearly time-optimal trajectories as the vehicle drives along the racing circuit in almost real-time. Our planning algorithm was able to consider stationary obstacles which were discovered using the sensors of the vehicle in previous laps.

We implemented two trajectory following strategies – the Pure Pursuit controller and the \gls{DWA} algorithm. Both algorithms enabled the vehicle to move along the planned trajectory, both with their own advantages. The Pure Pursuit algorithm follows the planned trajectory closely, but it follows it blindly and it does not avoid any unexpected obstacles. The \gls*{DWA} algorithm achieves similar comparable results and fast lap times and at the same time it allows the vehicle to avoid any obstacles which are detected as the vehicle moves and which were not considered by the planning algorithm.

We tested our algorithms on a car-like robot we built ourselves similarly to the F1/10 racing platform and on top of the \gls{ROS} ecosystem of tools and libraries. Although the planning algorithm performed sufficiently even with limited computation power, this test was not a success. Due to unreliable state estimation caused by the limited capabilities of the sensors, we were able to reach only very limited speeds at which we were able to move around the circuit. At higher speeds the vehicle would quickly lose track of its correct location and it would crash into a wall. When we eliminated the noisy data from the sensors and moved to a simulated environment, the vehicle was able to reach much higher speeds and reliably complete several laps of the circuit in a row.
nd Hybrid A* algorithms.

The experiments we conducted in a simulator show that our approach is viable. We are able to plan trajectories for the next three corners ahead of the vehicle at a frequency of several hertz and achieve good results on a test circuit. The most obvious shortcoming is the imprecise kinematic vehicle model. If we were able to predict the movement of the vehicle more reliably, the trajectories the planning algorithm produces would be easier to track and we would be able to navigate the vehicle through corners more safely without hitting the outer boundary of a corner. We could achieve further improvements by tuning the discretization parameters of the algorithms and the parameters and weights of the Pure Pursuit and DWA algorithms. It would be interesting to try the planning algorithm for a different F1/10 vehicle with reliable odometry and with identified parameters of a tire model. Since our code uses the standard \gls*{ROS} topic types and the standard transformation tree, the adaptation of our \gls*{ROS} nodes should be possible.