\chapter*{Conclusion}
\addcontentsline{toc}{chapter}{Conclusion}

In this thesis, we tried to solve the problem of autonomous racing inspired by the F1/10 competition. This problem is a simplification of full-scale autonomous vehicle driving which allowed us to focus solely on collision-free time-optimal car-like vehicle motion without having to take into account other intricacies of the adjacent problems, such as obeying traffic rules and predicting the behavior of other road users and pedestrians.

Our approach to solve the problem was to create an agent, which analyzes the layout of the racing circuit and then plans its motion through the track focusing on just a fixed number of corners ahead of the vehicle. Using the Hybrid A* and the \gls{SEHS} algorithms, we were able to generate nearly time-optimal trajectories as the vehicle drives along the racing circuit in almost real-time. Our planning algorithm was able to consider stationary obstacles which were discovered using the sensors of the vehicle in previous laps.

We implemented two trajectory following strategies – the Pure Pursuit controller and the \gls{DWA} algorithm. Both algorithms enabled the vehicle to move along the planned trajectory, both with their own advantages. We were able to achieve higher speeds and better lap times with the Pure Pursuit algorithm, but it was not able to avoid any unexpected obstacles. The \gls*{DWA} algorithm did not lead such good performance, but it allowed the vehicle to move along a track with occasional unexpected obstacles scattered across the track.

We tested our algorithms on a car-like robot we built ourselves similarly to the F1/10 racing platform and on top of the \gls{ROS} ecosystem of tools and libraries. Although the planning algorithm performed sufficiently even with limited computation power, this test was not a success. Due to unreliable state estimation caused by the limited capabilities of the sensors, we were able to reach only very limited speeds at which we were able to move around the circuit. At higher speeds the vehicle would quickly lose track of its correct location and it would crash into a wall. When we eliminated the noisy data from the sensors and moved to a simulated environment, the vehicle was able to reach much higher speeds and reliably complete several laps of the circuit in a row.

The experiments we conducted show that our approach is functional, but it is far from perfect and it could be improved in several ways. The most obvious shortcoming is the imprecise kinematic vehicle model. If we were able to predict the movement of the vehicle more reliably, the trajectories the planning algorithm produces would be easier to track and we would be able to navigate the vehicle through corners more safely without hitting the outer boundary of a corner. It would also be possible to tweak the roughness of the search space dimensions. We saw great improvement when we switched from the uniform grid discretization of the spatial dimensions of Hybrid A* to an adaptive path of circles of variable radii of \gls*{SEHS}. We might see further performance improvements by more finely tuning the discretization parameters of the longitudinal velocity and the heading angle. It would be interesting to try the planning algorithm for a different F1/10 vehicle with reliable odometry and with identified parameters of a tire model. Since our code uses the standard \gls*{ROS} topic types and the standard transformation tree, the adaptation of our \gls*{ROS} nodes should be straightforward.